\documentclass{article}
\usepackage{amsthm}
\usepackage{amsmath}
\usepackage{natbib}
\usepackage[colorlinks,citecolor=blue,urlcolor=blue,filecolor=blue,backref=page]{hyperref}
\usepackage{graphicx}
\usepackage{graphicx, psfrag, amsfonts, float}
\usepackage{natbib}
\usepackage{amsthm}
\usepackage{multirow}
\usepackage{hhline}
 \pdfminorversion=4
\usepackage{rotating}
\def\bgam{\mbox{\boldmath $\gamma$}}
\def\bth{\mbox{\boldmath $\theta$}}
\def\bbeta{\mbox{\boldmath $\beta$}}
\def\blam{\mbox{\boldmath $\lambda$}}
\def\bmu{\mbox{\boldmath $\mu$}}

\newcommand{\bx}{\mbox{\boldmath $x$}}
\newcommand{\bu}{\mbox{\boldmath $u$}}
\newcommand{\bc}{\mbox{\boldmath $c$}}
\newcommand{\bs}{\mbox{\boldmath $s$}}
\newcommand{\by}{\mbox{\boldmath $y$}}
\newcommand{\bz}{\mbox{\boldmath $z$}}
\newcommand{\bv}{\mbox{\boldmath $v$}}
\newcommand{\bb}{\mbox{\boldmath $b$}}
\newcommand{\bzero}{\mbox{\boldmath $0$}}
\newcommand{\thstarj}{\mbox{$\theta^\star_j$}}
\newcommand{\bths}{\mbox{$\btheta^\star$}}
\newcommand{\pkg}[1]{{\fontseries{b}\selectfont #1}} 

\newcommand{\ud}{\mathrm{d}}
\newcommand{\uI}{\mathrm{I}}
\newcommand{\uP}{\mathrm{P}}
\newcommand{\up}{\mathrm{p}}
\newcommand{\mb}{\mathbf}
\newcommand{\mc}{\mathcal}


\newcommand{\Bern}{\mbox{Bern}}
\newcommand{\Nor}{\mbox{N}}
\newcommand{\Ga}{\mbox{Gamma}}
\newcommand{\Dir}{\mbox{Dir}}
\newcommand{\Ber}{\mbox{Ber}}
\newcommand{\Be}{\mbox{Be}}
\newcommand{\Unif}{\mbox{Unif}}
\newcommand{\Binom}{\mbox{Bin}}
\newcommand{\IG}{\mbox{IG}}

\newcommand{\Xs}{X^\star}
\newcommand{\Lc}{{\cal L}}


\usepackage[dvipsnames,usenames]{color}
\newcommand{\yy}{\color{magenta}\it}
\newcommand{\jj}{\color{Black}\rm}
\newcommand{\yjnote}[1]{\footnote{\color{Brown}\rm #1 \color{Black}}}
\newtheorem{theorem}{Theorem}[section]
\newtheorem{definition}[theorem]{\bf Definition}
\newtheorem{lemma}[theorem]{\bf Lemma}
\newtheorem{corollary}[theorem]{\bf Corollary}
\newtheorem{proposition}[theorem]{\bf Proposition}
\newtheorem{assumption}[theorem]{\bf Assumption}
\newtheorem{example}[theorem]{\bf Example}
\newtheorem{remark}[theorem]{\bf Remark}


\newcommand{\iid}{\stackrel{iid}{\sim}}
\newcommand{\indep}{\stackrel{indep}{\sim}}


\setlength{\textwidth}{6 in}
\usepackage{color}
\usepackage{ulem} % \sout
\newcommand{\red}[1]{{\color{red}#1}}
\newcommand{\blue}[1]{{\color{blue}#1}}
\newcommand{\brown}[1]{{\color{brown}#1}}
\newcommand{\green}[1]{{\color{green}#1}}
\newcommand{\magenta}[1]{{\color{magenta}#1}}
\newcommand{\response}[1]{{\color{blue}#1}}
\DeclareMathOperator*{\argmin}{\arg\!\min}
\graphicspath{{figures/}{figs/}}




\begin{document}
Thank you for the additional comments on our paper. As with the earlier comments, they are interesting.  We have taken all of them seriously and have worked on them in putting the revision together.  For the revision itself, we have tried to keep our edits (and new material) focused to avoid adding too much to a paper that is already lengthy.    

The new parts for this round of edits appear in blue, including text that has been stricken (lined out in the text).  

Specifically, we have done the following:

\begin{itemize}
\item We have slightly restructured the introduction and added a quote from a vintage paper on robustness.  We hope that this serves to clarify a long-standing tradition in robust statistics--namely that the techniques address both model misspecification and outlyingness, though descriptions often fall back to the "code word" outliers.  We could add many more references here, but we went with a single one to avoid detracting from the focus of the paper.  The Belgian call data (Section 2.4) is decidedly in the camp of model misspecification.  

\item We have expanded commentary on the connection to ABC methods in the literature review in Section 2.3.  

\item We have coded up an ABC-MCMC method for the real application in Section 5.3.  The method was recommended to us by our local faculty (van Zandt and Turner) who work with ABC and MCMC methods. Perhaps surprisingly, our method is only a little (1.6 times) slower per iterate, yet appears to mix better than ABC-MCMC.  To our eyes, the results of the data analysis look better for restricted likelihood than for ABC-MCMC. 

\end{itemize}


\end{document}

